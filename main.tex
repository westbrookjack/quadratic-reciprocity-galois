% Make sure xcolor gets options before any package loads it:
\PassOptionsToPackage{dvipsnames}{xcolor}

\documentclass{article}

% ---------- Page geometry ----------
\usepackage{geometry}
\geometry{left=1.2in, right=1.2in, top=1.2in, bottom=1.2in}

% ---------- Encoding & language ----------
\usepackage[utf8]{inputenc}
\usepackage[english]{babel}

% ---------- Math & theorem tools ----------
\usepackage{mathtools}  % loads amsmath
\usepackage{amssymb}
\usepackage{amsthm}
\usepackage{stmaryrd}   % \llbracket \rrbracket
\usepackage{mathrsfs}   % \mathscr
\usepackage{dsfont}     % \mathds{1}

% ---------- Graphics / diagrams ----------
\usepackage{tikz}
\usetikzlibrary{cd}
\usetikzlibrary{shapes.geometric,arrows,positioning,fit,calc}
\usepackage[all]{xy}

% ---------- Lists, floats, tables, layout helpers ----------
\usepackage[shortlabels]{enumitem}
\usepackage{float}
\usepackage{multirow}
\usepackage{adjustbox}

% ---------- Colors (already has dvipsnames via PassOptionsToPackage) ----------
\usepackage{xcolor}

% ---------- Bibliography (BibLaTeX) ----------
\usepackage{csquotes}
\usepackage{biblatex}
\addbibresource{references.bib}

% ---------- Titles / TOC helpers ----------
\usepackage{titlesec}

\newcommand{\chapterheading}[1]{%
  \section*{#1}%
  \addcontentsline{toc}{section}{#1}%
}
\newcommand{\topic}[1]{%
  \phantomsection
  \subsection*{#1}%
  \addcontentsline{toc}{subsection}{#1}%
}
\newcommand{\subtopic}[1]{%
  \phantomsection
  \subsubsection*{#1}%
  \addcontentsline{toc}{subsubsection}{#1}%
}

% Custom theorem styles for “displayed statements”
\newtheoremstyle{customplain}{1em}{1em}{\itshape}{}{\bfseries}{.}{ }{}
\newtheoremstyle{customdef}{1em}{1em}{}{}{\bfseries}{.}{ }{}

\theoremstyle{customplain}
\newtheorem*{customthm}{Statement}
\newtheorem*{customlem}{Statement}
\newtheorem*{customprop}{Statement}
\newtheorem*{customcor}{Statement}

\theoremstyle{customdef}
\newtheorem*{customdefn}{Statement}
\newtheorem*{customex}{Statement}
\newtheorem*{customexer}{Statement}
\newtheorem*{customremark}{Statement}

\newcommand{\tocexercise}[2]{%
  \phantomsection
  \subsubsection*{#1}%
  \addcontentsline{toc}{subsubsection}{\texorpdfstring{#1}{Exercise}}%
  \begin{customexer}#2\end{customexer}
}
\newcommand{\toctheorem}[2]{%
  \phantomsection
  \subsubsection*{#1}%
  \addcontentsline{toc}{subsubsection}{\texorpdfstring{#1}{Theorem}}%
  \begin{customthm}#2\end{customthm}
}
\newcommand{\toclemma}[2]{%
  \phantomsection
  \subsubsection*{#1}%
  \addcontentsline{toc}{subsubsection}{\texorpdfstring{#1}{Lemma}}%
  \begin{customlem}#2\end{customlem}
}
\newcommand{\tocproposition}[2]{%
  \phantomsection
  \subsubsection*{#1}%
  \addcontentsline{toc}{subsubsection}{\texorpdfstring{#1}{Proposition}}%
  \begin{customprop}#2\end{customprop}
}
\newcommand{\tocexample}[2]{%
  \phantomsection
  \subsubsection*{#1}%
  \addcontentsline{toc}{subsubsection}{\texorpdfstring{#1}{Example}}%
  \begin{customex}#2\end{customex}
}

% ---------- Standard theorem environments ----------
\theoremstyle{definition}
\newtheorem{definition}{Definition}[section]
\newtheorem{theorem}{Theorem}[section]
\newtheorem{lemma}[theorem]{Lemma}
\newtheorem{corollary}{Corollary}[theorem]
\newtheorem*{remark}{Remark}
\newtheorem*{lemma*}{Lemma}
\newtheorem*{theorem*}{Theorem}
\newtheorem*{corollary*}{Corollary}
\renewcommand{\qedsymbol}{$\blacksquare$}

% ---------- Hyperlinks (after most packages; before todonotes) ----------
\usepackage[colorlinks=true, linkcolor=blue, citecolor=blue, urlcolor=blue]{hyperref}

% ---------- Optional TODO notes (after hyperref) ----------
\usepackage{xargs}
\usepackage[colorinlistoftodos,prependcaption,textsize=tiny]{todonotes}
\newcommandx{\unsure}[2][1=]{\todo[linecolor=red,backgroundcolor=red!25,bordercolor=red,#1]{#2}}
\newcommandx{\change}[2][1=]{\todo[linecolor=blue,backgroundcolor=blue!25,bordercolor=blue,#1]{#2}}
\newcommandx{\info}[2][1=]{\todo[linecolor=OliveGreen,backgroundcolor=OliveGreen!25,bordercolor=OliveGreen,#1]{#2}}
\newcommandx{\improvement}[2][1=]{\todo[linecolor=Plum,backgroundcolor=Plum!25,bordercolor=Plum,#1]{#2}}
\newcommandx{\thiswillnotshow}[2][1=]{\todo[disable,#1]{#2}}

% ---------- Small typography tweaks ----------
\renewcommand{\le}{\leqslant}
\renewcommand{\ge}{\geqslant}

% ---------- Handy delimiters ----------
\newcommand{\abs}[1]{\left| #1 \right|}
\newcommand{\norm}[1]{\left\| #1 \right\|}
\newcommand{\llp}{\mathopen{(\!(}}
\newcommand{\rrp}{\mathclose{)\!)}}
\newcommand{\llb}{\llbracket}
\newcommand{\rrb}{\rrbracket}

% ---------- Blackboard/calligraphic/script ----------
\newcommand{\A}{\mathbb{A}}
\newcommand{\B}{\mathbb{B}}
\newcommand{\C}{\mathbb{C}}
\newcommand{\D}{\mathbb{D}}
\newcommand{\F}{\mathbb{F}}
% --- redefine \H safely ---
\makeatletter
% --- keep the original Ø and Hungarian accent ---
\AtBeginDocument{%
  \let\Otext\O     
  \let\Haccent\H   

  % make \O and \H math-safe everywhere
  \DeclareRobustCommand{\O}{\ensuremath{\mathrm{O}}}
  \DeclareRobustCommand{\H}{\ensuremath{\mathbb{H}}}
}
\makeatother
\newcommand{\N}{\mathbb{N}}
\newcommand{\Q}{\mathbb{Q}}
\newcommand{\R}{\mathbb{R}}
\newcommand{\RP}{\mathbb{RP}}
\newcommand{\CP}{\mathbb{CP}}
\newcommand{\HP}{\mathbb{HP}}
\newcommand{\T}{\mathbb{T}}
\newcommand{\Z}{\mathbb{Z}}
\let\Section\S
\renewcommand{\S}{\mathbb{S}}
\let\Paragraph\P
\renewcommand{\P}{\mathbb{P}}

\newcommand{\calA}{\mathcal{A}}
\newcommand{\calB}{\mathcal{B}}
\newcommand{\calC}{\mathcal{C}}
\newcommand{\calE}{\mathcal{E}}
\newcommand{\calH}{\mathcal{H}}
\newcommand{\calK}{\mathcal{K}}
\newcommand{\calL}{\mathcal{L}}
\newcommand{\calM}{\mathcal{M}}
\newcommand{\calO}{\mathcal{O}}
\newcommand{\calP}{\mathcal{P}}
\newcommand{\calQ}{\mathcal{Q}}
\newcommand{\calR}{\mathcal{R}}

\newcommand{\fA}{\mathscr{A}}
\newcommand{\fB}{\mathscr{B}}
\newcommand{\fC}{\mathscr{C}}
\newcommand{\fD}{\mathscr{D}}
\newcommand{\fE}{\mathscr{E}}
\newcommand{\fF}{\mathscr{F}}
\newcommand{\fG}{\mathscr{G}}
\newcommand{\fH}{\mathscr{H}}
\newcommand{\fI}{\mathscr{I}}
\newcommand{\fJ}{\scri{J}}
\newcommand{\fO}{\mathscr{O}}
\newcommand{\fS}{\mathscr{S}}
\newcommand{\fT}{\mathscr{T}}

% ---------- Fraktur ----------
\newcommand{\frkA}{\mathfrak{A}}
\newcommand{\frkB}{\mathfrak{B}}
\newcommand{\frkS}{\mathfrak{S}}
\newcommand{\frka}{\mathfrak{a}}
\newcommand{\frkb}{\mathfrak{b}}
\newcommand{\frkc}{\mathfrak{c}}
\newcommand{\frkm}{\mathfrak{m}}
\newcommand{\frkn}{\mathfrak{n}}
\newcommand{\frkp}{\mathfrak{p}}
\newcommand{\frkq}{\mathfrak{q}}
\newcommand{\frkl}{\mathfrak{l}}
\newcommand{\frkN}{\mathfrak{N}}
\newcommand{\frkgl}{\mathfrak{gl}}
\newcommand{\frksl}{\mathfrak{sl}}
\newcommand{\frkso}{\mathfrak{so}}
\newcommand{\frksp}{\mathfrak{sp}}
\newcommand{\frku}{\mathfrak{u}}
\newcommand{\frkg}{\mathfrak{g}}
\newcommand{\frkh}{\mathfrak{h}}

% ---------- Misc symbols ----------
\newcommand{\rddots}{\reflectbox{$\ddots$}}
\newcommand{\altid}{\mathds{1}}
\newcommand{\eps}{\varepsilon}
\newcommand\interior[1]{{#1}^{\circ}}
\newcommand{\ctd}{\Rightarrow \Leftarrow}
\newcommand{\actson}{\circlearrowright}
\providecommand\mapsfrom{\mathrel{\reflectbox{\ensuremath{\mapsto}}}} % (fix)

\newcommand{\sqdot}{\, \raisebox{0.5ex}{\scalebox{0.2}{$\blacksquare$}} \,}
\newcommand{\nsubset}{\not\subset}
\let\oldemptyset\emptyset
\let\emptyset\varnothing

% ---------- Number theory brackets ----------
\newcommand{\leg}[2]{\genfrac{(}{)}{}{}{#1}{#2}}
\newcommand{\jac}[2]{\genfrac{(}{)}{}{}{#1}{#2}}
\newcommand{\kron}[2]{\genfrac{(}{)}{}{}{#1}{#2}}

% ---------- Operators ----------
\DeclareMathOperator{\Ab}{\mathbf{Ab}}
\DeclareMathOperator{\Grp}{\mathbf{Grp}}
\DeclareMathOperator{\Ring}{\mathbf{Ring}}
\DeclareMathOperator{\CRing}{\mathbf{CRing}}
\DeclareMathOperator{\Rng}{\mathbf{Rng}}
\DeclareMathOperator{\Set}{\mathbf{Set}}
\DeclareMathOperator{\pSet}{\mathbf{Set}_{\bullet}}
\DeclareMathOperator{\Top}{\mathbf{Top}}
\DeclareMathOperator{\pTop}{\mathbf{Top}_{\bullet}}
\DeclareMathOperator{\Op}{\mathbf{Op}}
\DeclareMathOperator{\Vect}{\mathbf{Vect}}
\DeclareMathOperator{\Man}{\mathbf{Man}}
\DeclareMathOperator{\Mod}{\mathbf{Mod}}
\DeclareMathOperator{\Mon}{\mathbf{Mon}}
\DeclareMathOperator{\Cat}{\mathbf{Cat}}
\DeclareMathOperator{\Ssubset}{\mathbf{Subset}}
\DeclareMathOperator{\Com}{\mathbf{Com}}
\DeclareMathOperator{\Haus}{\mathbf{Haus}}
\DeclareMathOperator{\Comp}{\mathbf{Comp}}
\DeclareMathOperator{\Poset}{\mathbf{Poset}}
\DeclareMathOperator{\Graph}{\mathbf{Graph}}
\DeclareMathOperator{\Sch}{\mathbf{Sch}}
\DeclareMathOperator{\AffSch}{\mathbf{AffSch}}
\DeclareMathOperator{\Grph}{\mathbf{Grph}}
\DeclareMathOperator{\Rel}{\mathbf{Rel}}
\DeclareMathOperator{\CW}{\mathbf{CW}}
\DeclareMathOperator{\PreSh}{\mathbf{PreSh}}
\DeclareMathOperator{\Sh}{\mathbf{Sh}}
\DeclareMathOperator{\catD}{\mathbf{D}}
\DeclareMathOperator{\TopGrp}{\mathbf{TopGrp}}
\DeclareMathOperator{\Meas}{\mathbf{Meas}}
\DeclareMathOperator{\Cob}{\mathbf{Cob}}
\DeclareMathOperator{\LieAlg}{\mathbf{LieAlg}}
\DeclareMathOperator{\Ban}{\mathbf{Ban}}
\DeclareMathOperator{\Hilb}{\mathbf{Hilb}}
\DeclareMathOperator{\AlgC}{\mathbf{Alg_C}}
\DeclareMathOperator{\Rep}{\mathbf{Rep}}

\DeclareMathOperator{\res}{\mathrm{res}}
\DeclareMathOperator{\pre}{\mathrm{pre}}
\DeclareMathOperator{\ad}{\mathrm{ad}}
\DeclareMathOperator{\ord}{\mathrm{ord}}
\DeclareMathOperator{\diam}{\mathrm{diam}}
\DeclareMathOperator{\dist}{\mathrm{dist}}
\DeclareMathOperator{\meas}{\mathrm{meas}}
\DeclareMathOperator{\Vol}{\mathrm{Vol}}
\DeclareMathOperator{\Ind}{\mathrm{Ind}}
\DeclareMathOperator{\Res}{\mathrm{Res}}
\DeclareMathOperator{\End}{\mathrm{End}}
\DeclareMathOperator{\PGL}{\mathrm{PGL}}
\DeclareMathOperator{\Aff}{\mathrm{Aff}}
\DeclareMathOperator{\GL}{\mathrm{GL}}
\DeclareMathOperator{\SL}{\mathrm{SL}}
\DeclareMathOperator{\PSL}{\mathrm{PSL}}
\DeclareMathOperator{\U}{\mathrm{U}}
\DeclareMathOperator{\SO}{\mathrm{SO}}
\DeclareMathOperator{\SU}{\mathrm{SU}}
\DeclareMathOperator{\Sp}{\mathrm{Sp}}
\DeclareMathOperator{\Gal}{\mathrm{Gal}}
\DeclareMathOperator{\Stab}{\mathrm{Stab}}
\DeclareMathOperator{\Proj}{\mathrm{Proj}}
\DeclareMathOperator{\Div}{\mathrm{Div}}
\DeclareMathOperator{\Pic}{\mathrm{Pic}}
\DeclareMathOperator{\im}{\mathrm{im}}
\DeclareMathOperator{\coim}{\mathrm{coim}}
\DeclareMathOperator{\cok}{\mathrm{cok}}
\DeclareMathOperator{\colim}{\mathrm{colim}}
\DeclareMathOperator{\spn}{\mathrm{span}}
\DeclareMathOperator{\Sym}{\mathrm{Sym}}
\DeclareMathOperator{\Hom}{\mathrm{Hom}}
\DeclareMathOperator{\Mor}{\mathrm{Mor}}
\DeclareMathOperator{\Nat}{\mathrm{Nat}}
\DeclareMathOperator{\Tr}{\mathrm{Tr}}
\DeclareMathOperator{\trdeg}{\mathrm{tr.deg}}
\DeclareMathOperator{\Bd}{\mathrm{Bd}}
\DeclareMathOperator{\Ann}{\mathrm{Ann}}
\DeclareMathOperator{\Int}{\mathrm{Int}}
\DeclareMathOperator{\Homeo}{\mathrm{Homeo}}
\DeclareMathOperator{\Char}{\mathrm{char}}
\DeclareMathOperator{\nullity}{\mathrm{nullity}}
\DeclareMathOperator{\Aut}{\mathrm{Aut}}
\DeclareMathOperator{\Frac}{\mathrm{Frac}}
\DeclareMathOperator{\supp}{\mathrm{supp}}
\DeclareMathOperator{\rank}{\mathrm{rank}}
\DeclareMathOperator{\diag}{\mathrm{diag}}
\DeclareMathOperator{\sign}{\mathrm{sign}}
\DeclareMathOperator{\glue}{\mathrm{glue}}
\DeclareMathOperator{\kerpre}{\ker_{\text{pre}}}
\DeclareMathOperator{\cokpre}{\cok_{\text{pre}}}
\DeclareMathOperator{\impre}{\im_{\text{pre}}}
\DeclareMathOperator{\coimpre}{\coim_{\text{pre}}}
\DeclareMathOperator{\sh}{\mathrm{sh}}
\DeclareMathOperator{\ev}{\mathrm{ev}}
\DeclareMathOperator{\Spec}{\mathrm{Spec}}
\DeclareMathOperator{\Ext}{\mathrm{Ext}}
\DeclareMathOperator{\Tor}{\mathrm{Tor}}
\DeclareMathOperator{\lcm}{\mathrm{lcm}}
\DeclareMathOperator{\Nm}{\mathrm{Nm}}

% ---------- Extra arrows ----------
\makeatletter
\newcommand\xtwoheadrightarrow[2][]{%
  \ext@arrow 0579{\twoheadrightarrowfill@}{#1}{#2}}
\newcommand\twoheadrightarrowfill@{%
  \arrowfill@\relbar\relbar\twoheadrightarrow}
\makeatother

% ================================
% Standard arithmetic-geometry / Langlands macros
% ================================

% --- Global/local fields and completions ---
\newcommand{\Qbar}{\overline{\Q}}              % algebraic closure of Q
\newcommand{\Qp}{\Q_p}
\newcommand{\Qpbar}{\overline{\Q}_p}
\newcommand{\Cp}{\C_p}
\newcommand{\Zp}{\Z_p}
\newcommand{\Zhat}{\widehat{\Z}}
\newcommand{\Fp}{\F_p}
\newcommand{\Fq}{\F_q}
\newcommand{\Fbar}{\overline{\F}}
\newcommand{\Af}{\A_{\mathrm f}}               % finite adèles
\newcommand{\Aone}{\A^{1}}
\newcommand{\SpecZ}{\Spec \Z}
\newcommand{\SpecQ}{\Spec \Q}

% --- Rings of integers, valuations ---
\newcommand{\OO}{\mathcal{O}}
\newcommand{\OK}{\OO_K}
\newcommand{\OL}{\OO_L}
\newcommand{\mK}{\frkm_K}
\newcommand{\vp}{v_p}
\newcommand{\ordp}{\operatorname{ord}_p}
\newcommand{\ordv}{\operatorname{ord}_v}

% --- Galois groups & Frobenius ---
\newcommand{\GQ}{G_{\Q}}
\newcommand{\GQp}{G_{\Q_p}}
\newcommand{\GK}{G_{K}}
\newcommand{\GLQp}{\GL_2(\Qp)}
\newcommand{\Frob}{\mathrm{Frob}}
\newcommand{\Frobp}{\Frob_p}
\newcommand{\Frobv}{\Frob_v}
\newcommand{\Art}{\mathrm{Art}}                % Artin map

% --- Cohomology theories ---
\newcommand{\et}{\mathrm{\acute{e}t}}
\newcommand{\fppf}{\mathrm{fppf}}
\newcommand{\cris}{\mathrm{cris}}
\newcommand{\dR}{\mathrm{dR}}
\newcommand{\Het}{H_{\et}}
\newcommand{\Hc}{H_{\mathrm c}}
\newcommand{\RGamma}{\mathbf{R}\Gamma}
\newcommand{\Rfstar}{\mathbf{R}f_*}
\newcommand{\Lfstar}{\mathbf{L}f^*}

% --- p-adic Hodge theory ---
\newcommand{\Ainf}{A_{\mathrm{inf}}}
\newcommand{\BdR}{B_{\mathrm{dR}}}
\newcommand{\Bcris}{B_{\mathrm{cris}}}
\newcommand{\BHT}{B_{\mathrm{HT}}}

% --- Perfectoid/adic geometry ---
\newcommand{\Spa}{\operatorname{Spa}}
\newcommand{\Spf}{\operatorname{Spf}}
\newcommand{\Spm}{\operatorname{Spm}}
\newcommand{\tilt}{^{\flat}}                   % X^\flat

% --- Group schemes / stacks ---
\newcommand{\Gm}{\mathbb{G}_m}
\newcommand{\Ga}{\mathbb{G}_a}
\newcommand{\muN}[1]{\mu_{#1}}
\newcommand{\Bun}{\operatorname{Bun}}

% --- Elliptic curves & Diophantine notation ---
\newcommand{\E}{\mathcal{E}}
\newcommand{\Sel}{\mathrm{Sel}}

% Proper Tate–Shafarevich symbol (Cyrillic Ш)
\usepackage[T2A]{fontenc}
\newcommand{\Sha}{\text{\fontencoding{T2A}\selectfont\CYRSH}}

\newcommand{\heightN}{\hat{h}}                 % Néron–Tate height
\newcommand{\Tate}{\mathrm{Tate}}

% --- Modular forms, Hecke operators ---
\newcommand{\Mk}[1]{M_{#1}}
\newcommand{\Sk}[1]{S_{#1}}
\newcommand{\Tp}{T_p}
\newcommand{\Up}{U_p}
\newcommand{\GammaN}[1]{\Gamma_0(#1)}

% --- Norms, traces, pairings ---
\newcommand{\Nmk}[2]{\mathrm{Nm}_{#1/#2}}
\newcommand{\Trmk}[2]{\mathrm{Tr}_{#1/#2}}
\newcommand{\pair}[2]{\langle #1,\,#2\rangle}

% --- Category-theoretic & derived conventions ---
\newcommand{\Perf}{\operatorname{Perf}}
\newcommand{\QCoh}{\operatorname{QCoh}}
\newcommand{\Coh}{\operatorname{Coh}}
\newcommand{\Dperf}{\catD_{\mathrm{perf}}}
\newcommand{\Dqc}{\catD_{\mathrm{qc}}}
\newcommand{\Dcoh}{\catD_{\mathrm{coh}}}
\newcommand{\Vectf}{\Vect_{\mathrm f}}

% --- Arrow conveniences ---
\newcommand{\into}{\hookrightarrow}
\newcommand{\onto}{\twoheadrightarrow}
\newcommand{\isom}{\xrightarrow{\sim}}
\newcommand{\defeq}{\vcentcolon=}
\newcommand{\eqdef}{=\vcentcolon}
\newcommand{\tensorhat}{\widehat{\otimes}}

% --- Title Block ---
\makeatletter
\newcommand{\Course}[1]{\gdef\@Course{#1}}
\newcommand{\Instructor}[1]{\gdef\@Instructor{#1}}
\newcommand{\Collaborators}[1]{\gdef\@Collaborators{#1}}
\newcommand{\Student}[1]{\gdef\@Student{#1}}

% defaults so empties are safe
\gdef\@Course{}
\gdef\@Instructor{}
\gdef\@Collaborators{}
\gdef\@Student{Jack Westbrook}

% --- Custom maketitle ---
\renewcommand{\maketitle}{%
  \begin{center}
    {\Large\bfseries \@title\par}
    \vspace{0.5em}
    \ifx\@Course\@empty\else {\normalsize\textit{\@Course}\par}\fi
    \vspace{0.5em}
    {\normalsize \@Student\par}
    \ifx\@Instructor\@empty\else   {\small Instructor: \@Instructor\par}\fi
    \ifx\@Collaborators\@empty\else{\small Collaborators: \@Collaborators\par}\fi
    \vspace{0.5em}
    {\small \today}
  \end{center}
  \vspace{1em}%
}
\makeatother

% ---------- Title ----------
\title{A Proof of Quadratic Reciprocity by Galois Theory}
\Instructor{Prof. Ana Caraiani}
%\Collaborators{Alice Smith, Bob Johnson} % omit if none
% \Student{Jack Westbrook} % only if overriding the default

\begin{document}
\maketitle
\section*{}
Let $p,q$ be distinct, odd rational primes. Then
\[
\leg{q}{p}=(-1)^{\frac{p-1}{2} \frac{q-1}{2}} \leg{p}{q}.
\]
In addition,
\[
\leg{2}{p} = \begin{cases}
    1, & \text{ if } p \equiv \pm 1 \mod 8\\
    -1, & \text{ if } p \equiv \pm 3 \mod 8
\end{cases}
\]
and
\[
\leg{-1}{p}=(-1)^{\frac{p-1}{2}}.
\]

\begin{proof}
We let $\zeta_p$ denote a primitive $p$-th root of unity over $\Q$. We assume these basic results from algebraic number theory: $\Gal(\Q(\zeta_p)/\Q) \cong (\Z/p\Z)^\times$, and if $d$ is a squarefree integer then $\calO_{\Q(\sqrt{d})}=\Z[\alpha]$ where $\alpha = \begin{cases}
    \sqrt{d}, &\text{ if } d\equiv 2,3\mod 4\\
    \frac{1+\sqrt{d}}{2}, & \text{ if } d\equiv 1 \mod 4
\end{cases}$ and $\Delta(\Q(\sqrt{d}))=\begin{cases}
    d, & \text{ if } d \equiv 1 \mod 4\\
    4d, & \text{ if } d \equiv 2,3, \mod 4
\end{cases}$.

\begin{lemma}\label{lem:2 homs}
    For any group $G,$ $$\# \Hom(G, \Z/2\Z)=\#\{H\leq G \mid [G:H]=2\}+1.$$
\end{lemma}
\begin{proof}
    If $\phi:G\to \Z/2\Z$ is a nontrivial group homomorphism, then $\ker \phi$ is an index 2 subgroup of $G$. Moreover, if $\psi$ is another such homomorphism, then there is a homomorphism $\phi+\psi$ given by $(\phi+\psi)(x)=\phi(x)+\psi(x).$ Now if $\ker \phi = \ker \psi$, then $\phi+\psi=0$. To see this, divide into cases based on whether or not an arbitrary $x\in G$ is in $\ker \phi.$ As a consequence, $\phi+\phi=0$. 
    Then
    \[
    \phi  = (\phi+\psi)+\phi = (\phi+\phi)+\psi=\psi.
    \]
    Therefore the set of nonzero homomorphisms injects into the set of index 2 subgroups. In addition, since every index 2 subgroup is normal, every index 2 subgroup is the kernel of a homomorphism $G\to \Z/2\Z.$ Thus the set of nonzero homomorphisms is in bijection with the set of index 2 subgroups, giving the result.
\end{proof}

\begin{lemma}\label{lem:quad subfield}
    The unique quadratic subfield of $\Q(\zeta_p)$ is $\Q(\sqrt{\hat{p}})$ where $\hat{p}\defeq (-1)^{\frac{p-1}{2}} p.$
\end{lemma}
\begin{proof}
We see that $\Q(\zeta_p)$ has a unique quadratic subfield $\Q(\sqrt{d})$ (for some $d\in \Z$ squarefree) because there is a unique subgroup of order 2 in $\Gal(\Q(\zeta_p)/\Q).$ More is true; because $p$ is the only prime that ramifies in $\Q(\zeta_p)$ by Proposition 6.2 in \cite{milneANT}, it follows that $p$ is the only prime that ramifies in $\Q(\sqrt d)$ as well, so the discriminant must be a power of $p$ (positive or negative). However, the discriminant of $\Q(\sqrt{d})$ is $d$ if $d\equiv 1 \mod 4$ and is $4d$ otherwise. From this it follows that $d\equiv 1 \mod 4$ and that $d=\pm p.$ In particular, the quadratic subfield of $\Q(\zeta_p)$ is
\[
\begin{cases}
    \Q(\sqrt{p}), & \text{ if } p\equiv 1 \mod 4\\
    \Q(\sqrt{-p}), & \text{ if } p \equiv 3 \mod 4
\end{cases}
= \Q(\hat p).
\]
\end{proof}

First, we prove the easy result that $\leg{-1}{p}=(-1)^{\frac{p-1}{2}}.$ One way to do this is by considering the field $\F_p$, and an algebraically closed field $\Omega$ containing $\F_p.$ We notice that
\[
\F_p = \{x\in \Omega \mid x^p = x\}
\]
since the containment $\subset$ is clear, and both are sets of size $p$, so they must be equal. From this we deduce
\[
\F_p^\times=\{x\in \Omega \mid x^{p-1}=1\}.
\]
Now let $x\in \F_p^\times$ be arbitrary, and $y\in \Omega$ such that $y^2=x.$ We see that $\leg{x}{p}=1$ iff $y\in \F_p$ iff $y^{p-1}=1.$ Since $y^{p-1}=x^{\frac{p-1}{2}}\in \{\pm 1\}$, we get the exact sequence of groups
\[
1\to (\F_p^\times)^2 \to \F_p^\times \xrightarrow{x\mapsto x^{\frac{p-1}{2}}} \{\pm 1\} \to 1.
\]
Thus we deduce more generally that for any $x\in \F_p^\times$, $\leg{x}{p}= x^{\frac{p-1}{2}}$ (identifying $\{\pm 1\}$ with the copy embedded in $\F_p$).

Now to the problem of reciprocity. From Lemma~\ref{lem:quad subfield}, we have a nontrivial map $\res:\Gal(\Q(\zeta_p)/\Q)\to \Gal(\Q(\sqrt{\hat{p}})/\Q)$ (nontrivial since $\Q(\zeta_p)$ is not a quadratic extension), a map $\leg{\cdot}{p}:(\Z/p\Z)^\times \to \{\pm 1\}$, and isomorphisms $\Gal(\Q(\zeta_p)/\Q)\isom (\Z/p\Z)^\times$, $\Gal(\Q(\sqrt{\hat{p}})/\Q)\isom \{\pm 1\}.$ Because each composite is nontrivial, Lemma~\ref{lem:2 homs}, combined with the fact that for every cyclic group of order $n$ and $d\mid n$ has a unique subgroup of order $d$, gives that the below diagram of groups commutes:
\begin{center}
    \begin{tikzcd}
        \Gal(\Q(\zeta_p)/\Q) \ar{d}{\sim} \ar[two heads]{r}{\res} & \Gal(\Q(\sqrt{\hat p})/\Q) \ar{d}{\sim}\\
        (\Z/p\Z)^\times \ar[two heads]{r}{\leg{\cdot}{p}} & \{\pm 1\}.
    \end{tikzcd}
\end{center}
There exists a unique ($\Frob$ always exists uniquely up to conjugacy, but since $\Q(\zeta_p)/\Q$ is abelian conjugacy classes are the same as elements) element $\Frob_q\in \Gal(\Q(\zeta_p)/\Q)$ such that for a (equivalently, any, since $\Q(\zeta_p)$ is abelian) $\Q(\zeta_p)$-prime $\frkq$ lying over $q$, $\Frob_q$ acts as $x\mapsto x^q $ on $\calO_{\Q(\zeta_p)}/\frkq $. In particular, since $\calO_{\Q(\zeta_p)}=\Z[\zeta_p]$ by Proposition 6.2 in \cite{milneANT}, and there exists an automorphism defined by $\zeta_p\mapsto \zeta_p^q$ in $\Gal(\Q(\zeta_p)/\Q)$ which visibly acts as $x\mapsto x^q$ on $\Z[\zeta_p]/\frkq$ for any prime $\frkq \mid q$, we see that $\Frob_q$ is the automorphism given by $\zeta_p\mapsto \zeta_p^q.$ The result will follow by tracking $\Frob_q$ around both sides of the diagram.

For the top side, since $q$ is unramified in $\Q(\sqrt{\hat{p}})$, it follows that either $q$ splits or is inert. Let $\frkq$ be a $\Q(\sqrt{\hat p})$-prime lying over $q.$ If $q$ splits, it follows (from $\sum_{\frkB \mid \frkp} e_\frkB f_\frkB = [L:K]$, i.e., Theorem 3.34 in \cite{milneANT}) that $\calO_{\Q(\sqrt{\hat{p}})}/\frkq \cong \F_q$, in which case $\res (\Frob_q) $ is indeed trivial on the residue field, hence trivial by uniqueness of $\Frob_q$ in $\Gal(\Q(\sqrt{\hat p})/\Q).$ On the other hand, if $q$ is inert, then $\calO_{\Q(\sqrt{\hat{p}})}/\frkq \cong \F_{q^2}$, in which case $x\mapsto x^q$ is nontrivial, so $\res(\Frob_q)$ is the nontrivial element of $\Gal(\Q(\sqrt{\hat p})/\Q).$

From Theorem 3.41 in \cite{milneANT}, $q$ splits if and only if $x^2-x+\frac{1-\hat p}{4}$ (the minimal polynomial of $\frac{1+\sqrt{\hat p}}{2}$, where $\calO_{\Q(\sqrt{\hat p})}=\Z[\frac{1+\sqrt{\hat p}}{2}]$) has a root modulo $q$. Because this polynomial has discriminant $\hat p$, this is if and only if $\leg{\hat p}{q}=1$ (any quadratic over a field of characteristic not 2 has a root if and only if its discriminant is a square). Putting these results together, we deduce that $\res(\Frob_q)\mapsto \leg{\hat p}{q}.$

For the other map to $\{\pm 1\}$, we see $\Frob_q\mapsto q\mapsto \leg{q}{p}.$

Thus commutativity gives
\[
\leg{q}{p}=\leg{(-1)^{\frac{p-1}{2}} p}{q} = \leg{-1}{q}^{\frac{p-1}{2}} \leg{p}{q} = \left((-1)^{\frac{q-1}{2}}\right)^{\frac{p-1}{2}} \leg{p}{q}=(-1)^{\frac{p-1}{2}\frac{q-1}{2}}\leg{p}{q}.
\]

For the last claim, we will track $\Frob_2$, which again is defined by $\zeta_p\mapsto \zeta_p^2.$ Like before, $2$ is either split or inert in $\Q(\sqrt{\hat p}).$ From Theorem 3.41 in \cite{milneANT}, we know that since $\calO_{\Q(\sqrt{\hat p})} = \Z[\alpha]$ where $\alpha = \frac{1+\sqrt{\hat p}}{2}$ with minimal polynomial $f(x)=x^2-x+\frac{1-\hat p}{4}$, then $2$ splits if and only if $f(x)$ is irreducible modulo $2.$ If $\hat p\equiv 1 \mod 8$, i.e., $\frac{1-\hat p}{4}$ is even, 
\[
f(x)\equiv x(x+1) \mod 2
\]
so $2$ splits.
If instead $\hat p \equiv 5\mod 8$, i.e., $\frac{1-\hat p}{4}$ is odd, \[
f(x)\equiv x^2+x+1 \mod 2
\]
which is irreducible, so $2$ is inert.

We also notice that $\hat p \equiv 1 \mod 8$ if and only if $p \equiv \pm 1 \mod 8$, so by the same commutativity argument as before, we get that
\[
\leg{2}{p} =1 \iff p \equiv \pm 1 \mod 8
\]
which is the desired result.
\end{proof}


\printbibliography
\end{document}